%%%%%%%%%%%%%%%%%%%%%%%%%%%%%%%%%%%%%%%%%%%%%%%%%%%%%%%%%%%
\documentclass[xcolor=x11names,compress]{beamer}
%\documentclass[xcolor=x11names,compress, handhouts, aspectratio=169]{beamer}
%% General document
\usepackage{graphicx, subfig}
%% Beamer Layout
\useoutertheme[subsection=false,shadow]{miniframes}
\useinnertheme{default}
\usefonttheme{serif}
\usepackage{palatino}

%%%%%%% Mes Packages %%%%%%%%%%%%%%%%
%\usepackage[french]{babel}
\usepackage[T1]{fontenc}
\usepackage{color}
\usepackage{xcolor}
\usepackage{dsfont} % Pour indicatrice
\usepackage{url}
\usepackage{multirow}
\usepackage[normalem]{ulem}   % For strike out text

% Natbib for clean bibliography
\usepackage[comma,authoryear]{natbib}

%remove the icon
\setbeamertemplate{bibliography item}{}

%remove line breaks
\setbeamertemplate{bibliography entry title}{}
\setbeamertemplate{bibliography entry location}{}
\setbeamertemplate{bibliography entry note}{}

%% ------ MEs couleurs --------
\definecolor{vert}{rgb}{0.1,0.7,0.2}
\definecolor{brique}{rgb}{0.7,0.16,0.16}
\definecolor{gris}{rgb}{0.7, 0.75, 0.71}
\definecolor{twitterblue}{rgb}{0, 0.42, 0.58}
\definecolor{airforceblue}{rgb}{0.36, 0.54, 0.66}
\definecolor{siap}{RGB}{3,133, 200}


%%%%%%%%%%%%%%%%% BEAMER PACKAGE %%%%%%%

\setbeamercolor{itemize item}{fg=siap}
%\setbeamercolor{itemize subitem}{fg=blue}
%\setbeamercolor{itemize subsubitem}{fg=cyan}

\setbeamerfont{title like}{shape=\scshape}
\setbeamerfont{frametitle}{shape=\scshape}

\setbeamercolor*{lower separation line head}{bg=DeepSkyBlue4}
\setbeamercolor*{normal text}{fg=black,bg=white}
\setbeamercolor*{alerted text}{fg=siap}
\setbeamercolor*{example text}{fg=black}
\setbeamercolor*{structure}{fg=black}
\setbeamercolor*{palette tertiary}{fg=black,bg=black!10}
\setbeamercolor*{palette quaternary}{fg=black,bg=black!10}

% Set the header color to SIAP's color
\setbeamercolor*{frametitle}{fg=siap}

%remove navigation symbols
\setbeamertemplate{navigation symbols}{}

\renewcommand{\(}{\begin{columns}}
\renewcommand{\)}{\end{columns}}
\newcommand{\<}[1]{\begin{column}{#1}}
\renewcommand{\>}{\end{column}}

%% Add footer with logo
\setbeamertemplate{footline}{%
  \begin{beamercolorbox}[wd=\paperwidth,ht=2.5ex,dp=1.125ex,%
    leftskip=.3cm,rightskip=.3cm plus1fil]{author in head/foot}
    \includegraphics[height=4ex]{SIAP_logo_Big.png}\hfill
    \insertshortauthor\hfill\insertshorttitle\hfill  \textcolor{siap}{\textit{\insertframenumber}}
  \end{beamercolorbox}%
}

% Path for the graphs
\graphicspath{
{Graphics/}
{c:/Chris/UN-ESCAP/SIAP-E-learning/Resources/OpenScience/}
{c:/Chris/Visualisation/Presentations/Graphics/}
{c:/Chris/Visualisation/Presentations/Graphics/SIAP/}
{c:/Chris/Visualisation/Presentations/Graphics/Lies/}
{c:/Chris/Visualisation/Presentations/Graphics/Maps/}
{c:/Chris/Visualisation/Presentations/Graphics/RGenerated/}
{c:/Chris/Visualisation/Presentations/Graphics/Logos/}
{c:/Gitmain/MLCourse/UNML/Module0/M0_files/figure-html/}
{c:/Chris/UN-ESCAP/MyCourses2022/MLOS2022/Slides/Graphics/}
{c:/Chris/UN-ESCAP/MyCourses2023/RAP/Slides/Graphics/}
{c:/GitMain/RAP/RAP-Course/images/}
{c:/Chris/UN-ESCAP/MyCourses2022/MLOS2022/Slides/Graphics/}
% Path for specific graphs created
 {../R-Codes/JobSatisfaction_files/figure-latex/}
 {../R-Codes/Unused_files/figure-latex/}
 }


\title[\textcolor{siap}{Principles of RAP}]{\textcolor{siap}{Principles of \\ Reproducible Analytical Pipelines \\}
\vspace{0.55cm} \textcolor{brique}{How to work with Git in RStudio}}
\author{Christophe Bontemps}
\institute{\large{\emph{Statistical Institute for Asia and the Pacific} } \\
    \includegraphics[height=10ex]{SIAP_logo_Big.png}}
\date{}


\begin{document}

\begin{frame}
\titlepage
\end{frame}



%%%%%%%%%%%%%%%%%%%%%%%%%%%%%%%%% AI %%%%%%

\section{What is Version Control?}

\begin{frame} % Cover slide
\frametitle{Version Control keeps tracks of your work}
Tracking three \textbf{W} questions:
\begin{columns}[t]
 \begin{column}{0.5\textwidth}
 \begin{itemize}[<+->]
 \item[] \textcolor{siap}{\textbf{W}}hich changes?
 \item[] \textcolor{siap}{\textbf{W}}ho changed it?
 \item[] \textcolor{siap}{\textbf{W}}hen was the change?
 \end{itemize}
 \end{column}
  \begin{column}{0.5\textwidth}
    \begin{center}
    \begin{itemize}
        \only<1-4>{ \includegraphics[width=0.95\textwidth]{FileHistory.PNG} \\  }
       \only<1-4>{\hfill  \textcolor{gris}{\tiny{Source: \href{https://the-turing-way.netlify.app/reproducible-research/vcs.html}{The Turing Way project}}}}
    \end{itemize}
    \end{center}
  \end{column}
\end{columns}
\end{frame}


\begin{frame}
\frametitle{Projects are complex, memory is limited}
\begin{columns}[t]
 \begin{column}{0.6\textwidth}
 \begin{itemize}[<+->]
        \item Files, code, data evolves over time
        \item[$\hookrightarrow$] Who remember all the changes?
        \item Keeping track can be hard \& boring!
        \item Manual methods prove very inefficient
        \item[$\hookrightarrow$] Your future-self and team members may be lost

    \end{itemize}
 \end{column}
  \begin{column}{0.4\textwidth}
    \begin{center}
    \begin{itemize}
        \only<1-3>{ \includegraphics[width=0.95\textwidth]{ExempleGentzkow_Library_Bordel2.JPG} \\  }
       \only<1-3>{ \textcolor{gris}{\tiny{ Source: \href{https://web.stanford.edu/~gentzkow/research/CodeAndData.pdf}{Gentzkow and Shapiro (2014)}}} \\ }

        \only<4-5>{ \includegraphics[width=0.95\textwidth]{AnatomyOfaFile.png} \\  }
       \only<4-5>{ \textcolor{gris}{\tiny{ Source: \href{https://chazhutton.substack.com/p/making-my-life-harder}{Chaz Hutton}}}}
    \end{itemize}
    \end{center}
  \end{column}
\end{columns}
\end{frame}

\begin{frame}{Transparency, Accountability \& Reproducibility}
    \begin{itemize}[<+->]
        \item Version control provides a detailed history of changes
        \item Each modification is attributed to a specific user
        \item Promotes accountability, transparency \&  reproducibility
    \end{itemize}
\end{frame}

\section{File evolution}

\begin{frame}{The evolution of a file}
\begin{center}
\begin{itemize}
   \only<1> {\includegraphics[width = 1.0\textwidth]{FileChange1a.png} \\ }
   \only<2> {\includegraphics[width = 1.0\textwidth]{FileChange2a.png} \\ }
   \only<3> {\includegraphics[width = 1.0\textwidth]{FileChange3a.png} \\ }
   \only<4> {\includegraphics[width = 1.0\textwidth]{FileChange4a.png} \\ }
   \only<5> {\includegraphics[width = 1.0\textwidth]{FileChange5a.png} \\ }
   \only<6> {\includegraphics[width = 1.0\textwidth]{FileChange6a.png} \\ }
   \only<7> {\includegraphics[width = 1.0\textwidth]{FileChange7a.png} \\ }
\end{itemize}
\end{center}
\end{frame}

\begin{frame}{The evolution of a file}
Usual ways to keep track of changes:
\pause
\begin{columns}[t]
\begin{column}{0.6\textwidth}
\begin{itemize}[<+->]
        \item New file after each change
        \item[$\hookrightarrow$] Need to open each file to see the change
        \item[$\hookrightarrow$] Names have to be explicit
        \item Only the last file with lots of comments
        \item Not fulfilling the 3 \textcolor{siap}{\textbf{W}}...

    \end{itemize}
 \end{column}
  \begin{column}{0.4\textwidth}
    \begin{center}
    \begin{itemize}
        \only<1-4>{ \includegraphics[width=1.0\textwidth]{FileLifeAll.png} \\  }
        \only<5-6>{ \includegraphics[width=0.95\textwidth]{FileLifeFinalFinal.png} \\  }
       % \only<7>{ \includegraphics[width=1.0\textwidth]{FileLifeAll.png} \\  }
       % \only<7>{\hfill \includegraphics[width=0.5\textwidth]{FileLifeFinalFinal.png} \\  }

    \end{itemize}
    \end{center}
  \end{column}
\end{columns}
\end{frame}

\section{Version control}

\begin{frame}{The evolution \textcolor{brique}{with Version Control}}
Record a message (\emph{commit})  for each change!
\begin{center}
\begin{itemize}
   \only<1> {\includegraphics[width = 1.0\textwidth]{FileLife1.png} \\ }
   \only<2> {\includegraphics[width = 1.0\textwidth]{FileLife2.png} \\ }
   \only<3> {\includegraphics[width = 1.0\textwidth]{FileLife3.png} \\ }
   \only<4> {\includegraphics[width = 1.0\textwidth]{FileLife4.png} \\ }
   \only<5> {\includegraphics[width = 1.0\textwidth]{FileLife5.png} \\ }
   \only<6> {\includegraphics[width = 1.0\textwidth]{FileLife6.png} \\ }
   \only<7> {\includegraphics[width = 1.0\textwidth]{FileLife7.png} \\ }
\end{itemize}
\end{center}
\end{frame}

\begin{frame}{The history of the file is recorded!}
\begin{center}
\begin{itemize}
   \only<1-2> {Each version is documented (with \emph{commits}) \\ }
   \only<1-2> {\includegraphics[width = 1.0\textwidth]{FileLifeHistoryEnd.png} \\ }
   \only<3-4> {Each version embeds the full history!  }
   \only<3-4> {\includegraphics[width = 1.0\textwidth]{FileLifeHistoryFull.png} \\ }
\end{itemize}
\end{center}
\end{frame}

\begin{frame}{Going back and "undo" is possible}
\begin{center}
\begin{itemize}
   \only<1-2> {It is possible to review previous version... \\ }
   \only<1-2> {\includegraphics[width = 1.0\textwidth]{FileLifeBack.png} \\ }
   \only<3-4> {...to compare the changes...  }
   \only<3-4> {\includegraphics[width = 1.0\textwidth]{FileLifeDiff.png} \\ }
   \only<5-6> {... and to revert  to a  previous version... }
   \only<5-6> {\includegraphics[width = 1.0\textwidth]{FileLifeRevert.png} \\ }
   \only<7-8> {... or \emph{undo} as if nothing happened }
   \only<7-8> {\includegraphics[width = 1.0\textwidth]{FileLifeRevert2.png} \\ }

\end{itemize}
\end{center}
\end{frame}

\section{Complexity}

\begin{frame}{Good and Bad News}
Real life is more complex:
\pause
\begin{columns}[t]
\begin{column}{0.6\textwidth}
\begin{itemize}[<+->]
        \item[\textbf{+}] Version Control is integrated in RStudio
        \item[$\hookrightarrow$] Simple operations are easy
        \item[\textbf{+}] Collaborate on a project
        \item[$\hookrightarrow$] Track changes of others
        \item[\textbf{-}] Git is a bit  ``\emph{unfriendly}'' \href{https://git-scm.com/}{\includegraphics[width=0.1\textwidth]{GitLogo.png}}
        \item[$\hookrightarrow$] Complex situations appear easily
        \item[\textbf{-}] Git works mostly in command mode
    \end{itemize}
 \end{column}
  \begin{column}{0.4\textwidth}
    \begin{center}
    \begin{itemize}
        \only<1-3>{ \includegraphics[width=0.95\textwidth]{Rstudio-git-screenshot.png} \\  }
        \only<1-3>{\textcolor{gris}{\tiny{\href{https://happygitwithr.com/rstudio-see-git}{"\emph{Happy Git and GitHub for the useR}" (Jennifer Bryan)}}}}
        \only<4-5>{ \includegraphics[width=0.95\textwidth]{GitCloneSmall.png} \\  }
        \only<6-7>{ \includegraphics[width=0.6\textwidth]{GitError1.png} \\
                    \hfill \includegraphics[width=0.6\textwidth]{GitError2.png} \\
                     \includegraphics[width=0.6\textwidth]{GitError3.png} \\ }
         \only<8>{ \includegraphics[width=0.9\textwidth]{GitPrompt.png} \\  }
         \only<8>{ \textcolor{gris}{\tiny{ \href{https://www.c-sharpcorner.com/article/git-for-absolute-beginners-with-command-line-interface/}{Ashish Vishwakarma}}}}

    \end{itemize}
    \end{center}
  \end{column}
\end{columns}
\end{frame}

\section{Conclusion}

\begin{frame}{Takeaways}
Version control system:
\begin{columns}[t]
\begin{column}{0.6\textwidth}
\begin{itemize}[<+->]
    \item Allows to travel back in time
    \item Keeps track of all changes
    \item Allows to "undo" at any point
    \item Allows reviewing stages of development
    \item Allow collaborating on projects
    \item Backups your work
  \end{itemize}
 \end{column}
  \begin{column}{0.4\textwidth}
    \begin{center}
    \begin{itemize}
        \only<1-6>{\includegraphics[width=0.8\textwidth]{github-mark.png} }
        \only<1-6>{ \textcolor{gris}{\tiny{ \href{https://www.c-sharpcorner.com/article/git-for-absolute-beginners-with-command-line-interface/}{GitHub logo}}}}
    \end{itemize}
    \end{center}
  \end{column}
\end{columns}
\end{frame}


\begin{frame}{Takeaways}
\pause

\begin{itemize}[<+->]
    \item Version control is very useful \includegraphics[width=0.05\textwidth]{github-mark.png}
    \item Version control requires patience, training and experience
    \item Version control is essential for transparency, and reproducibility of any project
    \item Installing Git within RStudio can be tedious but worth it!
    \item There is guidance, tutorials and helping blogs...
\end{itemize}
\end{frame}




\end{document}


%%%%%%%%%%%%%%% Last Slide %%%%%%%%%%%%%%%%

\begin{frame}[allowframebreaks]%in case more than 1 slide needed
\frametitle{References}
    {\footnotesize
    %\bibliographystyle{authordate1}
    \bibliographystyle{apalike}
    \bibliography{c:/Chris/Visualisation/Visu}
    }
\end{frame}


%\bibliographystyle{authordate1}
%\bibliography{c:/Chris/Visualisation/Visu}
%\end{frame}

\begin{frame} % Cover slide
\frametitle{ }
\pause
 \begin{itemize}[<+->]
  \item[]
  \item
\end{itemize}
\end{frame}
