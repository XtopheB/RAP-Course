%%%%%%%%%%%%%%%%%%%%%%%%%%%%%%%%%%%%%%%%%%%%%%%%%%%%%%%%%%%
\documentclass[xcolor=x11names,compress]{beamer}
%\documentclass[xcolor=x11names,compress, handhouts, aspectratio=169]{beamer}
%% General document
\usepackage{graphicx, subfig}
%% Beamer Layout
\useoutertheme[subsection=false,shadow]{miniframes}
\useinnertheme{default}
\usefonttheme{serif}
\usepackage{palatino}

%%%%%%% Mes Packages %%%%%%%%%%%%%%%%
%\usepackage[french]{babel}
\usepackage[T1]{fontenc}
\usepackage{color}
\usepackage{xcolor}
\usepackage{dsfont} % Pour indicatrice
\usepackage{url}
\usepackage{multirow}
\usepackage[normalem]{ulem}   % For strike out text

% Natbib for clean bibliography
\usepackage[comma,authoryear]{natbib}

%remove the icon
\setbeamertemplate{bibliography item}{}

%remove line breaks
\setbeamertemplate{bibliography entry title}{}
\setbeamertemplate{bibliography entry location}{}
\setbeamertemplate{bibliography entry note}{}

%% ------ MEs couleurs --------
\definecolor{vert}{rgb}{0.1,0.7,0.2}
\definecolor{brique}{rgb}{0.7,0.16,0.16}
\definecolor{gris}{rgb}{0.7, 0.75, 0.71}
\definecolor{twitterblue}{rgb}{0, 0.42, 0.58}
\definecolor{airforceblue}{rgb}{0.36, 0.54, 0.66}
\definecolor{violet}{RGB}{112,48,160}
\definecolor{orange}{RGB}{233,113,50}
\definecolor{siap}{RGB}{3,133, 200}

 
 


%%%%%%%%%%%%%%%%% BEAMER PACKAGE %%%%%%%

\setbeamercolor{itemize item}{fg=siap}
%\setbeamercolor{itemize subitem}{fg=blue}
%\setbeamercolor{itemize subsubitem}{fg=cyan}

\setbeamerfont{title like}{shape=\scshape}
\setbeamerfont{frametitle}{shape=\scshape}

\setbeamercolor*{lower separation line head}{bg=DeepSkyBlue4}
\setbeamercolor*{normal text}{fg=black,bg=white}
\setbeamercolor*{alerted text}{fg=siap}
\setbeamercolor*{example text}{fg=black}
\setbeamercolor*{structure}{fg=black}
\setbeamercolor*{palette tertiary}{fg=black,bg=black!10}
\setbeamercolor*{palette quaternary}{fg=black,bg=black!10}

% Set the header color to SIAP's color
\setbeamercolor*{frametitle}{fg=siap}

%remove navigation symbols
\setbeamertemplate{navigation symbols}{}

\renewcommand{\(}{\begin{columns}}
\renewcommand{\)}{\end{columns}}
\newcommand{\<}[1]{\begin{column}{#1}}
\renewcommand{\>}{\end{column}}

%% Add footer with logo
\setbeamertemplate{footline}{%
  \begin{beamercolorbox}[wd=\paperwidth,ht=2.5ex,dp=1.125ex,%
    leftskip=.3cm,rightskip=.3cm plus1fil]{author in head/foot}
    \includegraphics[height=4ex]{SIAP_logo_2020_1800.png}\hfill
    \insertshortauthor\hfill\insertshorttitle\hfill  \textcolor{siap}{\textit{\insertframenumber}}
  \end{beamercolorbox}%
}

% Path for the graphs
\graphicspath{
{Graphics/}
{c:/Chris/UN-ESCAP/SIAP-E-learning/Resources/OpenScience/}
{c:/Chris/UN-ESCAP/MyCourses2023/RAP/Slides/Graphics/}
{c:/GitMain/RAP/RAP-Course/images/}
{c:/GitMain/RAP/RAP-Course/images/logos/}
% Path for specific graphs created
 }


\title[\textcolor{siap}{Principles of RAP}]{\textcolor{siap}{Principles of \\ Reproducible Analytical Pipelines \\}
\vspace{0.55cm} \textcolor{brique}{Git and GitHub}}
\author{Christophe Bontemps}
\institute{\large{\emph{Statistical Institute for Asia and the Pacific} } \\
    \includegraphics[height=10ex]{SIAP_logo_2020_1800.png}}
\date{}


\begin{document}

\begin{frame}
\titlepage
\end{frame}



%%%%%%%%%%%%%%%%%%%%%%%%%%%%%%%%% AI %%%%%%

\section{Version Control}

\begin{frame}
\frametitle{Towards a full RAP}
\begin{columns}[t]
 \begin{column}{0.8\textwidth}

 \begin{itemize}[<+->]
        \item In short,\\
         \begin{center} \textbf{RAP}\\ =\\ Reproducible documents\\+ \\ Version Control \end{center}
        \item New tools, new challenges, new problems...
        \item[$\hookrightarrow$]  Requires time, patience and training
        \item Git is powerful, but complex and...
        \item[]\hfill ... unfriendly!!
        \item[$\hookrightarrow$] How it works
        \item[$\hookrightarrow$] How it works with \includegraphics[width=6ex]{GitHub-logo.png} \&  \includegraphics[width=2.5ex]{RStudio_logo.png} 
    \end{itemize}
 \end{column}
  \begin{column}{0.2\textwidth}
    \begin{center}
    \begin{itemize}
        \only<4-7>{\vspace{2 cm}  \includegraphics[width=0.8\textwidth]{git_logo.jpg} \\  }
        %\only<4-5>{ \includegraphics[width=0.95\textwidth]{GitHub-logo.png} \\  }

    \end{itemize}
    \end{center}
  \end{column}
\end{columns}
\end{frame}

\begin{frame}
\frametitle{Git  \& GitHub}
\begin{center}
    Git \& GitHub are different tools
\end{center}
\begin{columns}[t]
 \begin{column}{0.5\textwidth}
 \begin{itemize}[<+->]
  \item[]{\begin{center}
              \includegraphics[width=0.2\textwidth]{git_logo.jpg} \\
             \end{center} }
        \item Git is a software
        \item Git needs to be installed
        \item Git works mostly in command mode
        \item Git is complex and unfriendly!

    \end{itemize}
 \end{column}
 \begin{column}{0.5\textwidth}
 \begin{itemize}[<+->]
  \item[]{\begin{center}
              \includegraphics[width=0.4\textwidth]{GitHub-logo.png} \\
             \end{center} }
        \item GitHub is a platform
        \item GitHub needs an account
        \item GitHub can only be accessed remotely
        \item GitHub is (more) friendly
    \end{itemize}
 \end{column}
\end{columns}
\end{frame}


\section{Setup}

\begin{frame}
\frametitle{Setup}

What you'll need to do...
\begin{columns}[t]
 \begin{column}{0.8\textwidth}

 \begin{itemize}[<+->]
        \item Download Git
        \item Install Git on your computer
        \item[$\hookrightarrow$]  Depends on your OS
        \item Create a  GitHub account
        \item Link Git with GitHub account
        \item Link RStudio with Git
        \item[$\hookrightarrow$] May need IT support
        \item[$\hookrightarrow$] Many tutorials exist
    \end{itemize}
 \end{column}
  \begin{column}{0.2\textwidth}
    \begin{center}
    \begin{itemize}
        \only<1-8>{ \includegraphics[width=0.6\textwidth]{git_logo.jpg} \\  }
        \only<4-8>{\vspace{0.1 cm} \includegraphics[width=0.95\textwidth]{GitHub-logo.png}  \\  }
        \only<6-8>{ \vspace{0.1cm}  \includegraphics[width=0.6\textwidth]{RStudio_logo.png} \\  }
    \end{itemize}
    \end{center}
  \end{column}
\end{columns}
\end{frame}

\section{Principles}

\begin{frame}
\frametitle{Principles in action - Overview}
\begin{itemize}
\item[] \textcolor{violet}{\textbf{Alice}} and \textcolor{orange}{\textbf{Bob}} work on the same file\\ \vspace{1cm}
    \only<1>{\includegraphics[width=1.1\textwidth]{GitSimple1.png} \\  }
    \only<2>{\includegraphics[width=1.1\textwidth]{GitSimple2.png} \\  }
    \only<3>{\includegraphics[width=1.1\textwidth]{GitSimple3.png} \\  }
    \only<4>{\includegraphics[width=1.1\textwidth]{GitSimple4.png}   }
\end{itemize}
\end{frame}

\begin{frame}
\frametitle{Important notions}
\begin{columns}[t]
 \begin{column}{0.5\textwidth}
    \begin{itemize}[<+->]
     \item[]
   \begin{center}
    \textcolor{brique}{\textbf{Spaces}}
   \end{center}
    \item  \textcolor{siap}{\textbf{Local working directory:}}
    \item[$\hookrightarrow$] Folder with your file(s)
    \item  \textcolor{siap}{\textbf{Staging area:}}
    \item[$\hookrightarrow$] Local "space" where modified file(s) are stored before commit
    \item \textcolor{siap}{\textbf{Remote repository:}}
    \item[$\hookrightarrow$] Folder  on GitHub platform
    \end{itemize}
 \end{column}
 \begin{column}{0.5\textwidth}
    \begin{itemize}[<+->]
        \item[] 
        \begin{center}
        \textcolor{brique}{\textbf{Actions}}
        \end{center}
        \item \textcolor{siap}{\textbf{Commit:}}
        \item[$\hookrightarrow$] A snapshot of file changes
        \item \textcolor{siap}{\textbf{Commit message:}}
        \item[$\hookrightarrow$] A concise description of the changes made
        \item \textcolor{siap}{\textbf{Push:}}
        \item[$\hookrightarrow$] Action of sending modified file(s) to GitHub
        \item \textcolor{siap}{\textbf{Pull:}}
        \item[$\hookrightarrow$] Action of retrieving modified file(s) to local working directory

        
    \end{itemize}
 \end{column}
\end{columns}
\end{frame}


\begin{frame}
\frametitle{Principles in action - Details}
\textcolor{violet}{\textbf{Alice}} and \textcolor{orange}{\textbf{Bob}} work on the same file (Details)
\pause
\begin{itemize}
\item[]
\only<1>{\includegraphics[width=1.0\textwidth]{GitPrinciple1.png} \\ }
\only<2>{\includegraphics[width=1.0\textwidth]{GitPrinciple2.png} \\ }
\only<3>{\includegraphics[width=1.0\textwidth]{GitPrinciple3.png} \\ }
\only<4>{\includegraphics[width=1.0\textwidth]{GitPrinciple4.png} \\ }
\only<5>{\includegraphics[width=1.0\textwidth]{GitPrinciple5.png} \\ }
\only<6>{\includegraphics[width=1.0\textwidth]{GitPrinciple6.png} \\ }
\only<7>{\includegraphics[width=1.0\textwidth]{GitPrinciple7.png} \\ }
\only<8>{\includegraphics[width=1.0\textwidth]{GitPrinciple8.png} \\ }
\only<9>{\includegraphics[width=1.0\textwidth]{GitPrinciple9.png} \\ }
\only<10>{\includegraphics[width=1.0\textwidth]{GitPrinciple10.png} \\ }
\only<11>{\includegraphics[width=1.0\textwidth]{GitPrinciple11.png} \\ }
\only<12>{\includegraphics[width=1.0\textwidth]{GitPrinciple12.png} \\ }
\only<13>{\includegraphics[width=1.0\textwidth]{GitPrinciple13.png} \\ }
\end{itemize}
\end{frame}

\section{Issues}

\begin{frame}
\frametitle{Remarks and issues}
    \begin{itemize}[<+->]
     \item A commit can include changes from multiple files simultaneously
     \item One can do several commits before pushing (to GitHub)
     \item Commit messages can be edited (\texttt{amend})
     \item Every commit has an identifier (\texttt{hash} or \texttt{SHA}) 
     \item Git manage complex situations
     \item Only a few actions available in RStudio
    \end{itemize}
\end{frame}

\begin{frame}
\frametitle{Principles in action - Complex situations}
\textcolor{violet}{\textbf{Alice}} and \textcolor{orange}{\textbf{Bob}} work on the same file \textbf{at the same time}
\pause
\begin{itemize}
\item[]
\only<1>{\includegraphics[width=1.0\textwidth]{GitComplex1.png} \\ }
\only<2>{\includegraphics[width=1.0\textwidth]{GitComplex2.png} \\ }
\only<3>{\includegraphics[width=1.0\textwidth]{GitComplex3.png} \\ }
\only<4>{\includegraphics[width=1.0\textwidth]{GitComplex4.png} \\ }
\only<5>{\includegraphics[width=1.0\textwidth]{GitComplex5.png} \\ }
\only<6>{\includegraphics[width=1.0\textwidth]{GitComplex6.png} \\ }
\only<7>{\includegraphics[width=1.0\textwidth]{GitComplex7.png} \\ }
\only<8>{\includegraphics[width=1.0\textwidth]{GitComplex8.png} \\ }
\only<9>{\includegraphics[width=1.0\textwidth]{GitComplex9.png} \\ }
%\only<10>{\includegraphics[width=1.0\textwidth]{GitComplex10.png} \\ }
\end{itemize}
\end{frame}


\section{Takeaways}

\begin{frame}
\frametitle{Takeaways}
 Git: 
\begin{itemize}[<+->]
    \item Tracks files changes over time
    \item[$\hookrightarrow$] Changes can be visualized
    \item Facilitates team collaboration  and file sharing
    \item Manages complex situations of simultaneous changes
    \item Works well with GitHub \& Rstudio
    \item[$\hookrightarrow$] Advanced operations need line command instructions \\
    \includegraphics[width=0.7\textwidth]{GitPrompt.png} \\  
    \end{itemize}
\end{frame}

\begin{frame}
\frametitle{To conclude}
\begin{columns}[t]
 \begin{column}{0.5\textwidth}
    \begin{itemize}[<+->]
    \item Git can be challenging!
    \item Learning Git takes time
    \item Simple operations easy to use
    \item[$\hookrightarrow$] Work in teams
   
    \end{itemize}
 \end{column}
 \begin{column}{0.5\textwidth}
    \begin{itemize}
    \item[]
    \only<1-5>{\includegraphics[width=0.6\textwidth]{GitError1.png} \\ }
    \only<1-5>{\hfill \includegraphics[width=0.6\textwidth]{GitError2.png} \\ }
    \only<1-5>{\includegraphics[width=0.6\textwidth]{GitError3.png} \\ }
    \end{itemize}
 \end{column}
\end{columns}
 \begin{itemize}[<+->]
 \item[] 
    \item[] 
    \begin{center}
    Installing/working with Git \\
     \textbf{is not mandatory} for this course
    \end{center}
  \end{itemize}   
\end{frame}


\end{document}


 \only<1>{\includegraphics[width=1.0\textwidth]{GitPrinciple1.png} \\  
 \only<2>{\includegraphics[width=1.0\textwidth]{GitPrinciple2.png} \\  
 \only<3>{\includegraphics[width=1.0\textwidth]{GitPrinciple3.png} \\  
 \only<4>{\includegraphics[width=1.0\textwidth]{GitPrinciple4.png} \\  
 \only<5>{\includegraphics[width=1.0\textwidth]{GitPrinciple5.png} \\  
 \only<6>{\includegraphics[width=1.0\textwidth]{GitPrinciple6.png} \\  
 \only<7>{\includegraphics[width=1.0\textwidth]{GitPrinciple7.png} \\  
 \only<8>{\includegraphics[width=1.0\textwidth]{GitPrinciple8.png} \\  
 \only<9>{\includegraphics[width=1.0\textwidth]{GitPrinciple9.png} \\  
 \only<10>{\includegraphics[width=1.0\textwidth]{GitPrinciple10.png} \\
 \only<11>{\includegraphics[width=1.0\textwidth]{GitPrinciple11.png} \\
 \only<12>{\includegraphics[width=1.0\textwidth]{GitPrinciple12.png} \\
 \only<13>{\includegraphics[width=1.0\textwidth]{GitPrinciple13.png} \\
 \only<14>{\includegraphics[width=1.0\textwidth]{GitPrinciple14.png} \\
 \only<15>{\includegraphics[width=1.0\textwidth]{GitPrinciple15.png} \\
 \only<16>{\includegraphics[width=1.0\textwidth]{GitPrinciple16.png} \\
 \only<17>{\includegraphics[width=1.0\textwidth]{GitPrinciple17.png} \\
 \only<18>{\includegraphics[width=1.0\textwidth]{GitPrinciple18.png} \\
 \only<19>{\includegraphics[width=1.0\textwidth]{GitPrinciple19.png} \\ 


\end{document}


%%%%%%%%%%%%%%% Last Slide %%%%%%%%%%%%%%%%

\begin{frame}[allowframebreaks]%in case more than 1 slide needed
\frametitle{References}
    {\footnotesize
    %\bibliographystyle{authordate1}
    \bibliographystyle{apalike}
    \bibliography{c:/Chris/Visualisation/Visu}
    }
\end{frame}


%\bibliographystyle{authordate1}
%\bibliography{c:/Chris/Visualisation/Visu}
%\end{frame}

\begin{frame} % Cover slide
\frametitle{ }
\pause
 \begin{itemize}[<+->]
  \item[]
  \item
\end{itemize}
\end{frame}
